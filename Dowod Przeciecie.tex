Weźmy dowolne $k \in \mathbb{W}$. Pokażemy, że $\mathbb{T}_k \cap \mathbb{NT}_k = \emptyset$.
Załóżmy nie wprost, że $\mathbb{T}_k \cap \mathbb{NT}_k \neq \emptyset$. Zatem istnieją takie $i, j$, że $i \neq j$ i $\mathbb{T}_{k, i} = \mathbb{NT}_{k, j}$. Zauważmy, że jeśli $\mathbb{T}_{k, i} = \mathbb{NT}_{k, j}$, to w szczególności każdy sufiks obu napisów jest taki sam. Z tego wynika, że sufiks długości 2 i 3 też są takie same dla obu napisów. Sprzeczność \lightning (bo sufiks długości 2 i 3 to podstawy, odpowiednio $\mathbb{T}_1$, $\mathbb{NT}_1$ i $\mathbb{T}_2$, $\mathbb{NT}_2$, których używamy do tworzenia kolejnych zbiorów, a one są różne)