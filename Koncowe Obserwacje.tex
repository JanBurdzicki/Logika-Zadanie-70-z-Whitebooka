Zauważmy, że powyższe rozwiązanie można sprowadzić do sprawdzania parzystości najdłuższego sufiksu, którego wszystkie elementy są takie same. Jeśli sufiks jest parzystej długości, to $(... (( p^{i_1} \Imp p^{i_2} ) \Imp p^{i_3}) \Imp ...) \Imp p^{i_n}$ jest tautologią, w przeciwnym przypadku $(... (( p^{i_1} \Imp p^{i_2} ) \Imp p^{i_3}) \Imp ...) \Imp p^{i_n}$ nie jest tautologią. Można to rozwiązanie udowodnić za pomocą prostej indukcji.