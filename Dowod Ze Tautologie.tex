Niech $\mathbb{X} \subseteq \mathbb{W}$ taki, że:
\begin{itemize}
	\item $1 \in \mathbb{X}$, oraz
	\item $2 \in \mathbb{X}$, oraz
	\item $3 \in \mathbb{X}$, oraz
	\item dla dowolnego $n \in \mathbb{W}$, $n \geq 2$, jeśli $n \in \mathbb{X}$ oraz $(n + 1) \in \mathbb{X}$, to $(n + 2) \in \mathbb{X}$
\end{itemize}
Wtedy $\mathbb{X} = \mathbb{W}$
\n

\noindent
Niech $\mathbb{X} = \{n \in \mathbb{W} | \mathbb{T}_n$ zawiera tylko napisy binarne reprezentujące tautologie$\}$
\n

\noindent
W dowodzie skorzystamy z powyższej zasady indukcji.
\n

\noindent
Dowód indukcyjny:
\n

\noindent
Podstawa indukcji:
\begin{itemize}
\item $1 \in \mathbb{X}$, bo $\mathbb{T}_1 = \emptyset$
\item $2 \in \mathbb{X}$, bo $\mathbb{T}_2 = \{00, 11\}$ (wynika to z naszych obserwacji $2.6 - 2.10$)
\item $3 \in \mathbb{X}$, bo $\mathbb{T}_3 = \{100, 011\}$ (wynika to z naszych obserwacji $2.6 - 2.10$)
\end{itemize}

\noindent
Krok indukcyjny:
\n
\noindent
Weźmy dowolne $n \in \mathbb{W}, n \geq 2$. Załóżmy, że $n \in \mathbb{X}$ i $(n + 1) \in \mathbb{X}$. Pokażemy, że $(n + 2) \in \mathbb{X}$.
\n

\noindent
Rozważmy wszystkie przypadki, w jaki sposób mogliśmy stworzyć i-ty napis ze zbioru $\mathbb{T}_{n + 2}$. Przy analizie będziemy posługiwali się naszymi obserwacjami $2.6 - 2.10$.
\begin{itemize}
	\item dodanie "przeciwnej cyfry":
	\begin{itemize}
		\item $\mathbb{T}_{n + 1, i\prime, 1} = 0$:

		Niech $\mathbb{T}_{n + 1, i\prime} = "0" + S$. Wtedy $\mathbb{T}_{n + 2, i} = "10" + S"$, czyli $\mathbb{T}_{n + 2, i} \equiv "0" + S = \mathbb{T}_{n + 1, i\prime}$. Skoro $\mathbb{T}_{n + 1, i\prime}$ jest napisem reprezentującym tautologię, to $\mathbb{T}_{n + 2, i}$ też jest.

		\item $\mathbb{T}_{n + 1, 1} = 1$:

		Niech $\mathbb{T}_{n + 1, i\prime} = "1" + S$. Wtedy $\mathbb{T}_{n + 2, i} = "01" + S"$, czyli $\mathbb{T}_{n + 2, i} \equiv "1" + S = \mathbb{T}_{n + 1, i\prime}$. Skoro $\mathbb{T}_{n + 1, i\prime}$ jest napisem reprezentującym tautologię, to $\mathbb{T}_{n + 2, i}$ też jest.
	\end{itemize}
	\item dodanie $"00"$:

	Wtedy $\mathbb{T}_{n + 2, i} = "00" + \mathbb{T}_{n, i\prime}"$, czyli $\mathbb{T}_{n + 2, i} \equiv \mathbb{T}_{n, i\prime}$. Skoro $\mathbb{T}_{n, i\prime}$ jest napisem reprezentującym tautologię, to $\mathbb{T}_{n + 2, i}$ też jest.
	\item dodanie $"11"$:

	Wtedy $\mathbb{T}_{n + 2, i} = "11" + \mathbb{T}_{n, i\prime}"$, czyli $\mathbb{T}_{n + 2, i} \equiv \mathbb{T}_{n, i\prime}$. Skoro $\mathbb{T}_{n, i\prime}$ jest napisem reprezentującym tautologię, to $\mathbb{T}_{n + 2, i}$ też jest.
\end{itemize}

Zatem $(n + 2) \in \mathbb{X}$.
\n

\noindent
To znaczy, że na mocy tw. o indukcji $\mathbb{X} = \mathbb{W}$, czyli $\mathbb{T}_k$ zawiera tylko napisy binarne reprezentujące tautologie.