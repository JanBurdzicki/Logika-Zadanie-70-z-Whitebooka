\documentclass{article}
\usepackage[T1]{fontenc}
\usepackage{authblk}
\usepackage{amsfonts}
\usepackage{amsmath, amssymb, amsthm}
\usepackage{wasysym}
\usepackage{enumitem} %to jest do robienia podpunktów
% \usepackage{marvosym}
% \Lightning
\def\Imp{\Rightarrow}
\def\n{\newline}
\def\sigmad{\hat{\sigma}}
\newcommand\pp[2]{(p^#1 \Imp p^#2)}

\title{\textbf{Zadanie 70. z Whitebooka}}
\author{\textbf{Jan Burdzicki}}
\affil{II UWr}
% \date{27 November 2022}
\date{\today}

\begin{document}

\maketitle

\vspace{50pt}

\renewcommand{\contentsname}{Spis treści}
\tableofcontents

\newpage

\section{Opis problemu}

Niech $p^0$ oznacza ${\neg}p$ oraz niech $p^1$ oznacza $p$. Dla jakich ciągów
\n
$(i_1, ... , i_n) \in \{0, 1\}^n$ formuła

\begin{center}
$(... ((p^{i_1} \Imp p^{i_2}) \Imp p^{i_3}) \Imp ...) \Imp p^{i_n}$
\end{center}
\noindent
jest tautologią?

\section{Opis notacji}

\begin{itemize}
	\item $p^0$ będziemy utożsamiać z $\neg p$
	\item $p^1$ będziemy utożsamiać z $p$
	\item ciągi postaci $(i_1, ... , i_k) \in \{0, 1\}^k$ będziemy utożsamiać z napisami długości k nad alfabetem binarnym
\end{itemize}

\section{Podstawowe obserwacje}

\input{Podstawowe obserwacje}

% \newpage

\section{Definicja i konstrukcja zbiorów}

\input{Definicja i konstrukcja zbiorow}
% \newpage

\section{Dowód, że rodzina zbiorów $\mathbb{T}_k$ zawiera tylko napisy binarne reprezentujące tautologie}

Niech $\mathbb{X} \subseteq \mathbb{W}$ taki, że:
\begin{itemize}
	\item $1 \in \mathbb{X}$, oraz
	\item $2 \in \mathbb{X}$, oraz
	\item $3 \in \mathbb{X}$, oraz
	\item dla dowolnego $n \in \mathbb{W}$, $n \geq 2$, jeśli $n \in \mathbb{X}$ oraz $(n + 1) \in \mathbb{X}$, to $(n + 2) \in \mathbb{X}$
\end{itemize}
Wtedy $\mathbb{X} = \mathbb{W}$
\n

\noindent
Niech $\mathbb{X} = \{n \in \mathbb{W} | \mathbb{T}_n$ zawiera tylko napisy binarne reprezentujące tautologie$\}$
\n

\noindent
W dowodzie skorzystamy z powyższej zasady indukcji.
\n

\noindent
Dowód indukcyjny:
\n

\noindent
Podstawa indukcji:
\begin{itemize}
\item $1 \in \mathbb{X}$, bo $\mathbb{T}_1 = \emptyset$
\item $2 \in \mathbb{X}$, bo $\mathbb{T}_2 = \{00, 11\}$ (wynika to z naszych obserwacji $2.6 - 2.10$)
\item $3 \in \mathbb{X}$, bo $\mathbb{T}_3 = \{100, 011\}$ (wynika to z naszych obserwacji $2.6 - 2.10$)
\end{itemize}

\noindent
Krok indukcyjny:
\n
\noindent
Weźmy dowolne $n \in \mathbb{W}, n \geq 2$. Załóżmy, że $n \in \mathbb{X}$ i $(n + 1) \in \mathbb{X}$. Pokażemy, że $(n + 2) \in \mathbb{X}$.
\n

\noindent
Rozważmy wszystkie przypadki, w jaki sposób mogliśmy stworzyć i-ty napis ze zbioru $\mathbb{T}_{n + 2}$. Przy analizie będziemy posługiwali się naszymi obserwacjami $2.6 - 2.10$.
\begin{itemize}
	\item dodanie "przeciwnej cyfry":
	\begin{itemize}
		\item $\mathbb{T}_{n + 1, i\prime, 1} = 0$:

		Niech $\mathbb{T}_{n + 1, i\prime} = "0" + S$. Wtedy $\mathbb{T}_{n + 2, i} = "10" + S"$, czyli $\mathbb{T}_{n + 2, i} \equiv "0" + S = \mathbb{T}_{n + 1, i\prime}$. Skoro $\mathbb{T}_{n + 1, i\prime}$ jest napisem reprezentującym tautologię, to $\mathbb{T}_{n + 2, i}$ też jest.

		\item $\mathbb{T}_{n + 1, 1} = 1$:

		Niech $\mathbb{T}_{n + 1, i\prime} = "1" + S$. Wtedy $\mathbb{T}_{n + 2, i} = "01" + S"$, czyli $\mathbb{T}_{n + 2, i} \equiv "1" + S = \mathbb{T}_{n + 1, i\prime}$. Skoro $\mathbb{T}_{n + 1, i\prime}$ jest napisem reprezentującym tautologię, to $\mathbb{T}_{n + 2, i}$ też jest.
	\end{itemize}
	\item dodanie $"00"$:

	Wtedy $\mathbb{T}_{n + 2, i} = "00" + \mathbb{T}_{n, i\prime}"$, czyli $\mathbb{T}_{n + 2, i} \equiv \mathbb{T}_{n, i\prime}$. Skoro $\mathbb{T}_{n, i\prime}$ jest napisem reprezentującym tautologię, to $\mathbb{T}_{n + 2, i}$ też jest.
	\item dodanie $"11"$:

	Wtedy $\mathbb{T}_{n + 2, i} = "11" + \mathbb{T}_{n, i\prime}"$, czyli $\mathbb{T}_{n + 2, i} \equiv \mathbb{T}_{n, i\prime}$. Skoro $\mathbb{T}_{n, i\prime}$ jest napisem reprezentującym tautologię, to $\mathbb{T}_{n + 2, i}$ też jest.
\end{itemize}

Zatem $(n + 2) \in \mathbb{X}$.
\n

\noindent
To znaczy, że na mocy tw. o indukcji $\mathbb{X} = \mathbb{W}$, czyli $\mathbb{T}_k$ zawiera tylko napisy binarne reprezentujące tautologie.

\section{Dowód, że rodzina zbiorów $\mathbb{NT}_k$ zawiera tylko napisy binarne reprezentujące nietautologie}

Niech $\mathbb{X} \subseteq \mathbb{W}$ taki, że:
\begin{itemize}
	\item $1 \in \mathbb{X}$, oraz
	\item $2 \in \mathbb{X}$, oraz
	\item $3 \in \mathbb{X}$, oraz
	\item dla dowolnego $n \in \mathbb{W}$, $n \geq 2$, jeśli $n \in \mathbb{X}$ oraz $(n + 1) \in \mathbb{X}$, to $(n + 2) \in \mathbb{X}$
\end{itemize}
Wtedy $\mathbb{X} = \mathbb{W}$
\n

\noindent
Niech $\mathbb{X} = \{n \in \mathbb{W} | \mathbb{NT}_n$ zawiera tylko napisy binarne reprezentujące nietautologie$\}$
\n

\noindent
W dowodzie skorzystamy z powyższej zasady indukcji.
\n

\noindent
Dowód indukcyjny:
\n

\noindent
Podstawa indukcji:
\begin{itemize}
\item $1 \in \mathbb{X}$, bo $\mathbb{NT}_1 = \{0, 1\}$
\item $2 \in \mathbb{X}$, bo $\mathbb{NT}_2 = \{01, 10\}$ (wynika to z naszych obserwacji $2.6 - 2.10$)
\item $3 \in \mathbb{X}$, bo $\mathbb{NT}_3 = \{101, 010, 000, 001, 110, 111\}$ (wynika to z naszych obserwacji $2.6 - 2.10$)
\end{itemize}

\noindent
Krok indukcyjny:
\n
\noindent
Weźmy dowolne $n \in \mathbb{W}, n \geq 2$. Załóżmy, że $n \in \mathbb{X}$ i $(n + 1) \in \mathbb{X}$. Pokażemy, że $(n + 2) \in \mathbb{X}$.
\n

\noindent
Rozważmy wszystkie przypadki, w jaki sposób mogliśmy stworzyć i-ty napis ze zbioru $\mathbb{NT}_{n + 2}$. Przy analizie będziemy posługiwali się naszymi obserwacjami $2.6 - 2.10$.
\begin{itemize}
	\item dodanie "przeciwnej cyfry":
	\begin{itemize}
		\item $\mathbb{NT}_{n + 1, i\prime, 1} = 0$:

		Niech $\mathbb{NT}_{n + 1, i\prime} = "0" + S$. Wtedy $\mathbb{NT}_{n + 2, i} = "10" + S"$, czyli $\mathbb{NT}_{n + 2, i} \equiv "0" + S = \mathbb{NT}_{n + 1, i\prime}$. Skoro $\mathbb{NT}_{n + 1, i\prime}$ jest napisem reprezentującym nietautologię, to $\mathbb{NT}_{n + 2, i}$ też jest.

		\item $\mathbb{NT}_{n + 1, 1} = 1$:

		Niech $\mathbb{NT}_{n + 1, i\prime} = "1" + S$. Wtedy $\mathbb{NT}_{n + 2, i} = "01" + S"$, czyli $\mathbb{NT}_{n + 2, i} \equiv "1" + S = \mathbb{NT}_{n + 1, i\prime}$. Skoro $\mathbb{NT}_{n + 1, i\prime}$ jest napisem reprezentującym nietautologię, to $\mathbb{NT}_{n + 2, i}$ też jest.
	\end{itemize}
	\item dodanie $"00"$:

	Wtedy $\mathbb{NT}_{n + 2, i} = "00" + \mathbb{NT}_{n, i\prime}"$, czyli $\mathbb{NT}_{n + 2, i} \equiv \mathbb{NT}_{n, i\prime}$. Skoro $\mathbb{NT}_{n, i\prime}$ jest napisem reprezentującym nietautologię, to $\mathbb{NT}_{n + 2, i}$ też jest.
	\item dodanie $"11"$:

	Wtedy $\mathbb{NT}_{n + 2, i} = "11" + \mathbb{NT}_{n, i\prime}"$, czyli $\mathbb{NT}_{n + 2, i} \equiv \mathbb{NT}_{n, i\prime}$. Skoro $\mathbb{NT}_{n, i\prime}$ jest napisem reprezentującym nietautologię, to $\mathbb{NT}_{n + 2, i}$ też jest.
\end{itemize}

Zatem $(n + 2) \in \mathbb{X}$.
\n

\noindent
To znaczy, że na mocy tw. o indukcji $\mathbb{X} = \mathbb{W}$, czyli $\mathbb{T}_k$ zawiera tylko napisy binarne reprezentujące nietautologie.

% \section{Dowody unikalności elementów zbiorów}

\section{Dowód braku występowania duplikatów w rodzinie zbiorów $\mathbb{T}_k$}

Niech $\mathbb{X} \subseteq \mathbb{N}$ taki, że:
\begin{itemize}
	\item $1 \in \mathbb{X}$, oraz
	\item $2 \in \mathbb{X}$, oraz
	\item $3 \in \mathbb{X}$, oraz
	\item dla dowolnego $n \in \mathbb{N}$, $n \geq 2$, jeśli $n \in \mathbb{X}$ oraz $(n + 1) \in \mathbb{X}$, to $(n + 2) \in \mathbb{X}$
\end{itemize}
Wtedy $\mathbb{X} = \mathbb{N}$
\n

\noindent
Niech $\mathbb{X} = \{n \in \mathbb{N} | \mathbb{T}_n$ zawiera unikalne wartości$\}$
\n

\noindent
W dowodzie skorzystamy z powyższej zasady indukcji.
\n

\noindent
Dowód indukcyjny:
\n

\noindent
Podstawa indukcji:
\begin{itemize}
\item $1 \in \mathbb{X}$, bo $\mathbb{T}_1 = \emptyset$
\item $2 \in \mathbb{X}$, bo $\mathbb{T}_2 = \{00, 11\}$
\item $3 \in \mathbb{X}$, bo $\mathbb{T}_3 = \{100, 011\}$
\end{itemize}

\noindent
Krok indukcyjny:
\n
\noindent
Weźmy dowolne $n \in \mathbb{N}, n \geq 2$. Załóżmy, że $n \in \mathbb{X}$ i $(n + 1) \in \mathbb{X}$. Pokażemy, że $(n + 2) \in \mathbb{X}$.
\n

\noindent
Załóżmy nie wprost, że $\mathbb{T}_{n + 2}$ zawiera duplikaty. Zatem istnieją takie $i, j$, że $i \neq j$ i $\mathbb{T}_{n + 2, i} = \mathbb{T}_{n + 2, j}$. Rozważmy wszystkie przypadki, w jaki sposób mogliśmy stworzyć elementy $\mathbb{T}_{n + 2, i}$ i $\mathbb{T}_{n + 2, j}$:

\begin{itemize}
	\item Elementy $\mathbb{T}_{n + 2, i}$, $\mathbb{T}_{n + 2, j}$ powstały z elementów zbioru $\mathbb{T}_n$ poprzez dodanie $"00"$ lub $"11"$. Zauważmy, że skoro $\mathbb{T}_{n + 2, i} = \mathbb{T}_{n + 2, j}$, to w całości są takie same, więc w szczególności ostatnia operacja była taka sama i elementy, z których powstały elementy $\mathbb{T}_{n + 2, i}$ i $\mathbb{T}_{n + 2, j}$ też były. Czyli istnieją $i\prime, j\prime$ takie, że $i\prime \neq j\prime$ i $\mathbb{T}_{n, i\prime} = \mathbb{T}_{n, j\prime}$. Sprzeczność \lightning (bo z założenia indukcyjnego $\mathbb{T}_n \in X$, czyli zbiór $\mathbb{T}_n$ nie zawiera duplikatów).

	\item Elementy $\mathbb{T}_{n + 2, i}$, $\mathbb{T}_{n + 2, j}$ powstały z elementów zbioru $\mathbb{T}_{n + 1}$ poprzez dodanie "przeciwnej cyfry". Zauważmy, że skoro $\mathbb{T}_{n + 2, i} = \mathbb{T}_{n + 2, j}$, to w całości są takie same, więc w szczególności ostatnia operacja była taka sama i elementy, z których powstały elementy $\mathbb{T}_{n + 2, i}$ i $\mathbb{T}_{n + 2, j}$ też były. Czyli istnieją $i\prime, j\prime$ takie, że $i\prime \neq j\prime$ i $\mathbb{T}_{n + 1, i\prime} = \mathbb{T}_{n + 1, j\prime}$. Sprzeczność \lightning (bo z założenia indukcyjnego $\mathbb{T}_{n + 1} \in X$, czyli zbiór $\mathbb{T}_{n + 1}$ nie zawiera duplikatów).

	\item BSO: Załóżmy, że $\mathbb{T}_{n + 2, i}$ został stworzony z elementu zbioru $\mathbb{T}_n$, a $\mathbb{T}_{n + 2, j}$ został stworzony z elementu zbioru $\mathbb{T}_{n + 1}$. Zauważmy, że skoro $\mathbb{T}_{n + 2, i} = \mathbb{T}_{n + 2, j}$, to w całości są takie same, więc w szczególności pierwsze 2 znaki są takie same. Sprzeczność \lightning (bo pierwsze 2 znaki $\mathbb{T}_{n + 2, i} \in \{"00", "11"\}$, a  pierwsze 2 znaki $\mathbb{T}_{n + 2, j} \in \{"01", "10"\}$).
\end{itemize}

\noindent
We wszystkich przypadkach otrzymaliśmy sprzeczność, zatem $(n + 2) \in \mathbb{X}$.
\n

\noindent
To znaczy, że na mocy tw. o indukcji $\mathbb{X} = \mathbb{N}$, czyli $\mathbb{T}_n$ zawiera unikalne wartości.


\section{Dowód braku występowania duplikatów w rodzinie zbiorów $\mathbb{NT}_k$}

Niech $\mathbb{X} \subseteq \mathbb{N}$ taki, że:
\begin{itemize}
	\item $1 \in \mathbb{X}$, oraz
	\item $2 \in \mathbb{X}$, oraz
	\item $3 \in \mathbb{X}$, oraz
	\item dla dowolnego $n \in \mathbb{N}$, $n \geq 2$, jeśli $n \in \mathbb{X}$ oraz $(n + 1) \in \mathbb{X}$, to $(n + 2) \in \mathbb{X}$
\end{itemize}
Wtedy $\mathbb{X} = \mathbb{N}$
\n

\noindent
Niech $\mathbb{X} = \{n \in \mathbb{N} | \mathbb{NT}_n$ zawiera unikalne wartości$\}$
\n

\noindent
W dowodzie skorzystamy z powyższej zasady indukcji.
\n

\noindent
Dowód indukcyjny:
\n

\noindent
Podstawa indukcji:
\begin{itemize}
\item $1 \in \mathbb{X}$, bo $\mathbb{NT}_1 = \{0, 1\}$
\item $2 \in \mathbb{X}$, bo $\mathbb{NT}_2 = \{01, 10\}$
\item $3 \in \mathbb{X}$, bo $\mathbb{NT}_3 = \{101, 010, 000, 001, 110, 111\}$
\end{itemize}

\noindent
Krok indukcyjny:
\n
\noindent
Weźmy dowolne $n \in \mathbb{N}, n \geq 2$. Załóżmy, że $n \in \mathbb{X}$ i $(n + 1) \in \mathbb{X}$. Pokażemy, że $(n + 2) \in \mathbb{X}$.
\n

\noindent
Załóżmy nie wprost, że $\mathbb{NT}_{n + 2}$ zawiera duplikaty. Zatem istnieją takie $i, j$, że $i \neq j$ i $\mathbb{NT}_{n + 2, i} = \mathbb{NT}_{n + 2, j}$. Rozważmy wszystkie przypadki, w jaki sposób mogliśmy stworzyć elementy $\mathbb{NT}_{n + 2, i}$ i $\mathbb{NT}_{n + 2, j}$:

\begin{itemize}
	\item Elementy $\mathbb{NT}_{n + 2, i}$, $\mathbb{NT}_{n + 2, j}$ powstały z elementów zbioru $\mathbb{NT}_n$ poprzez dodanie $"00"$ lub $"11"$. Zauważmy, że skoro $\mathbb{NT}_{n + 2, i} = \mathbb{NT}_{n + 2, j}$, to w całości są takie same, więc w szczególności ostatnia operacja była taka sama i elementy, z których powstały elementy $\mathbb{NT}_{n + 2, i}$ i $\mathbb{NT}_{n + 2, j}$ też były. Czyli istnieją $i\prime, j\prime$ takie, że $i\prime \neq j\prime$ i $\mathbb{NT}_{n, i\prime} = \mathbb{NT}_{n, j\prime}$. Sprzeczność \lightning (bo z założenia indukcyjnego $\mathbb{NT}_n \in X$, czyli zbiór $\mathbb{NT}_n$ nie zawiera duplikatów).

	\item Elementy $\mathbb{NT}_{n + 2, i}$, $\mathbb{NT}_{n + 2, j}$ powstały z elementów zbioru $\mathbb{NT}_{n + 1}$ poprzez dodanie "przeciwnej cyfry". Zauważmy, że skoro $\mathbb{NT}_{n + 2, i} = \mathbb{NT}_{n + 2, j}$, to w całości są takie same, więc w szczególności ostatnia operacja była taka sama i elementy, z których powstały elementy $\mathbb{NT}_{n + 2, i}$ i $\mathbb{NT}_{n + 2, j}$ też były. Czyli istnieją $i\prime, j\prime$ takie, że $i\prime \neq j\prime$ i $\mathbb{NT}_{n + 1, i\prime} = \mathbb{NT}_{n + 1, j\prime}$. Sprzeczność \lightning (bo z założenia indukcyjnego $\mathbb{NT}_{n + 1} \in X$, czyli zbiór $\mathbb{NT}_{n + 1}$ nie zawiera duplikatów).

	\item BSO: Załóżmy, że $\mathbb{NT}_{n + 2, i}$ został stworzony z elementu zbioru $\mathbb{NT}_n$, a $\mathbb{NT}_{n + 2, j}$ został stworzony z elementu zbioru $\mathbb{NT}_{n + 1}$. Zauważmy, że skoro $\mathbb{NT}_{n + 2, i} = \mathbb{NT}_{n + 2, j}$, to w całości są takie same, więc w szczególności pierwsze 2 znaki są takie same. Sprzeczność \lightning (bo pierwsze 2 znaki $\mathbb{NT}_{n + 2, i} \in \{"00", "11"\}$, a  pierwsze 2 znaki $\mathbb{NT}_{n + 2, j} \in \{"01", "10"\}$).
\end{itemize}

\noindent
We wszystkich przypadkach otrzymaliśmy sprzeczność, zatem $(n + 2) \in \mathbb{X}$.
\n

\noindent
To znaczy, że na mocy tw. o indukcji $\mathbb{X} = \mathbb{N}$, czyli $\mathbb{NT}_n$ zawiera unikalne wartości.



\section{Dowód $\mathbb{T}_k \cap \mathbb{NT}_k = \emptyset$}

Weźmy dowolne $k \in \mathbb{W}$. Pokażemy, że $\mathbb{T}_k \cap \mathbb{NT}_k = \emptyset$.
Załóżmy nie wprost, że $\mathbb{T}_k \cap \mathbb{NT}_k \neq \emptyset$. Zatem istnieją takie $i, j$, że $i \neq j$ i $\mathbb{T}_{k, i} = \mathbb{NT}_{k, j}$. Zauważmy, że jeśli $\mathbb{T}_{k, i} = \mathbb{NT}_{k, j}$, to w szczególności każdy sufiks obu napisów jest taki sam. Z tego wynika, że sufiks długości 2 i 3 też są takie same dla obu napisów. Sprzeczność \lightning (bo sufiks długości 2 i 3 to podstawy, odpowiednio $\mathbb{T}_1$, $\mathbb{NT}_1$ i $\mathbb{T}_2$, $\mathbb{NT}_2$, których używamy do tworzenia kolejnych zbiorów, a one są różne)

\section{Dowód $\mathbb{T}_k \cup \mathbb{NT}_k$ zawiera wszystkie możliwe napisy długości k nad alfabetem binarnym}

Niech $R(X)$ oznacza funkcję zwracającą rozmiar zbioru $X$.
\n

\noindent
Zauważmy, że wszystkich możliwych napisów długości k nad alfabetem binarnym jest $2^k$.
\n

\noindent
Pokażemy, że $R(\mathbb{T}_k) + R(\mathbb{NT}_k) = 2^k$.
\n

\noindent
Niech $\mathbb{X} \subseteq \mathbb{N}$ taki, że:
\begin{itemize}
	\item $1 \in \mathbb{X}$, oraz
	\item $2 \in \mathbb{X}$, oraz
	\item dla dowolnego $n \in \mathbb{N}$, jeśli $n \in \mathbb{X}$ oraz $(n + 1) \in \mathbb{X}$, to $(n + 2) \in \mathbb{X}$
\end{itemize}
Wtedy $\mathbb{X} = \mathbb{N}$
\n

\noindent
Niech $\mathbb{X} = \{n \in \mathbb{N} | R(\mathbb{T}_n) + R(\mathbb{NT}_n) = 2^n\}$
\n

\noindent
W dowodzie skorzystamy z powyższej zasady indukcji.
\n

\noindent
Dowód indukcyjny:
\n

\noindent
Podstawa indukcji:
\begin{itemize}
\item $1 \in \mathbb{X}$, bo $R(\mathbb{T}_1) + R(\mathbb{NT}_1) = 0 + 2 = 2 = 2^1$
\item $2 \in \mathbb{X}$, bo $R(\mathbb{T}_2) + R(\mathbb{NT}_2) = 2 + 2 = 4 = 2^2$
\end{itemize}

\noindent
Krok indukcyjny:
\n
\noindent
Weźmy dowolne $n \in \mathbb{N}$. Załóżmy, że $n \in \mathbb{X}$ i $(n + 1) \in \mathbb{X}$. Pokażemy, że $(n + 2) \in \mathbb{X}$.
\n

\noindent
$R(\mathbb{T}_{n + 2}) + R(\mathbb{NT}_{n + 2}) = (2 * R(\mathbb{T}_{n}) + R(\mathbb{T}_{n + 1})) + (2 * R(\mathbb{NT}_{n}) + R(\mathbb{NT}_{n + 1})) = 2 * (R(\mathbb{T}_n) + R(\mathbb{NT}_n)) + (R(\mathbb{T}_{n + 1}) + R(\mathbb{NT}_{n + 1}))  \stackrel{\text{zał. ind.}}{=} 2 * 2^n + 2^{n + 1} = 2^{n + 1} + 2^{n + 1} = 2 * 2^{n + 1} = 2^{n + 2}$
\n

\noindent
Zatem $(n + 2) \in \mathbb{X}$.
\n

\noindent
To znaczy, że na mocy tw. o indukcji $\mathbb{X} = \mathbb{N}$, czyli $R(\mathbb{T}_n) + R(\mathbb{NT}_n) = 2^n$


\section{Podsumowanie}

Zauważmy, że skoro:
\begin{itemize}
	\item rodzina zbiorów $\mathbb{T}_{k}$ zawiera tylko tautologie
	\item rodzina zbiorów $\mathbb{NT}_{k}$ zawiera tylko nietautologie
	\item $\mathbb{T}_k \cap \mathbb{NT}_k = \emptyset$, czyli zbiór $\mathbb{T}_{k}$ nie zawiera wspólnego elementu ze zbiorem $\mathbb{NT}_{k}$
	\item $\mathbb{T}_k \cup \mathbb{NT}_k$ zawiera wszystkie możliwe napisy długości k nad alfabetem binarnym
\end{itemize}
to $\mathbb{T}_{k}$ zawiera wszystkie tautologie.

\section{Końcowe obserwacje}

Zauważmy, że powyższe rozwiązanie można sprowadzić do sprawdzania parzystości najdłuższego sufiksu, którego wszystkie elementy są takie same. Jeśli sufiks jest parzystej długości, to $(... (( p^{i_1} \Imp p^{i_2} ) \Imp p^{i_3}) \Imp ...) \Imp p^{i_n}$ jest tautologią, w przeciwnym przypadku $(... (( p^{i_1} \Imp p^{i_2} ) \Imp p^{i_3}) \Imp ...) \Imp p^{i_n}$ nie jest tautologią. Można to rozwiązanie udowodnić za pomocą prostej indukcji.

\end{document}
